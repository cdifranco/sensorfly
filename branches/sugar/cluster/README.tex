\documentclass{article}
\usepackage[margin=1in]{geometry}
\usepackage[colorlinks]{hyperref}
\begin{document}
\begin{center}
\LARGE \bf{SugarTrail Code Introduction}\\
\end{center}

\section{Phase 1: Signature Test}
\begin{itemize}
\item Location: 
	\begin{itemize}
	\item Getting Data: \textbackslash sugar\textbackslash cluster\textbackslash reading\_testing
	\item Analysis: \textbackslash sugar\textbackslash cluster\textbackslash analysis\textbackslash lab\_signature
	\end{itemize}
\item Fig: 
	\begin{itemize}
	\item Stable reading for one node
	\item Disturbed reading for one node
	\item Relatively stable signature for one node
	\end{itemize}
\item Run: \textbackslash sugar\textbackslash cluster\textbackslash analysis\textbackslash lab\_signature\textbackslash sig\_analisis.m
\end{itemize}


\section{Phase 2: Distribution Table/Curve generate}
\begin{itemize}
\item Location: \textbackslash sugar\textbackslash cluster\textbackslash distribution
\item Fig: 
	\begin{itemize}
	\item	 Relation of ranging distance and real distance
	\end{itemize}
\item Run: 
\end{itemize}


\section{Phase 3: Hallway Test}
\begin{itemize}
\item Location: \textbackslash sugar\textbackslash cluster\textbackslash hallway\_simulation
\item Fig: 
	\begin{itemize}
	\item Cluster
	\item Navigation
	\item Base number and performance
	\item Cluster size threshold and performance
	\end{itemize}
\item Run: main.c
\end{itemize}


\section{Phase 4 : Lab(Field Test)}
\begin{itemize}
\item Location: \textbackslash sugar\textbackslash cluster\textbackslash lab\_field\_test
\item Fig: None
\item Run: main.c
\end{itemize}


\section{Phase 5: Lab(Database Test)}
\begin{itemize}
\item Location: \textbackslash sugar\textbackslash cluster\textbackslash lab\_database
\item Fig: 
	\begin{itemize}
	\item Clustering(path\_length = 10000) 
	\item Navigaion
	\end{itemize}
\item Run:
	\begin{itemize}
	\item p1\_collect\_readings.m (collecting data, done, don't run again)
	\item p2\_data\_process.m (generate proper data structure for later, done, don't run again)
	\item p3\_generate\_cluster.m 
	\item p4\_match\_area\_to\_cluter.m 
	\item p5\_testing.m
	\item p6\_kmeans.m
	\end{itemize}
\end{itemize}

\section{Phase 6: Supermarket Experiment}
\begin{itemize}
\item Location: \textbackslash sugar\textbackslash cluster\textbackslash supermarket\_exp
\item Supermarket layout: supermarket.skp
\item Data: 
	\begin{itemize}
	\item Raw Data: \textbackslash sugar\textbackslash cluster\textbackslash supermarket\_exp\textbackslash step3\textbackslash raw\_data\textbackslash
	\item \underline{\Large{Complete Data and Documentation:}}  \textbackslash sugar\textbackslash cluster\textbackslash supermarket\_exp\textbackslash step3\textbackslash complete\_data\textbackslash
	\end{itemize}
\item Fig:
	\begin{itemize}
	\item Compass Direction
	\item Clustering(path\_length = 10000) 
	\item Navigation
	\end{itemize}
\item Run: 
	\begin{itemize}
	\item p1\_collect\_readings.m (collecting data, done, don't run again)
	\item p2\_data\_process.m (generate proper data structure for later, done, don't run again)
	\item draw\_compass\_direction.m
	\item p3\_generate\_cluster.m 
	\item p4\_testing.m
	\end{itemize} 
\item Code for analysis: \textbackslash sugar\textbackslash cluster\textbackslash supermarket\_exp\textbackslash analysis
	\begin{itemize}
	\item .m files:
		\begin{itemize}
		\item angle\_convert.m : convert direction information into angle
		\item dijkstra.m : Dijkstra Algorithm
		\item direction\_convert.m : convert current reading into direction info based on the reading at that point
		\item draw\_cluster.m : draw the clusters, should be updated with translucent style soon
		\item draw\_compass\_direction.m : draw the magnetic field based on compass reading in supermarket
		\item get\_cluster\_sig.m : get the cluster which the signature belongs to, used to implemented by choosing the minimum eclidean distance, but with the using of sub-set in signature, we change it to use possibility also, but it does not work well with the later one, cannot select correct cluster often. 
		\item get\_next\_point.m : used in the "training part", when the training paths are generated, it will give out the next point in path, based on random walk and the data from supermarket.
		\item get\_next\_step.m : used in the "testing part", when the guiding paths are generated, it will give out the next step location in navigation, based on guiding info and the data from supermarket.
		\item guide.m : the main part of guiding
		\item is\_blocked.m : check whether if the path of current point and next point is blocked by environment.
		\item navigate.m : called in testing.m for navigation from one point to another
		\item possibility.m : give out the possibility of one ready belongs to certain cluster.
		\item valid\_sig.m : check whether the signature reading is valid and how many valid info is in one signature. 
		\end{itemize}
	\item .mat files:
		\begin{itemize}
		\item 2feet\_grid.mat : real supermarket readings, ranging with different anchors and compass reading at each point of the grid
		\item processed\_data.mat : the data processed from raw data.
		\item distribution\_table.mat : calcutate the possibility of one reading belongs to the cluster with certain center
		\item environment\_info.mat : racks' edges info, for obstacle detection
		\end{itemize}
	\end{itemize}
\end{itemize}


\end{document}
